\documentclass{beamer}
\usepackage[english]{layout}
\usepackage[utf8]{inputenc}
\usepackage[english]{babel}
\usepackage[T1]{fontenc}
\usepackage{lmodern}
\usepackage{textcomp}
\usepackage{amsmath, soul, color, multicol, type1cm, verbatim, latexsym, dsfont, float, listings,alltt}
\usepackage[official]{eurosym}
\usepackage{beamerthemesplit}
\usetheme{Frankfurt}
\usefonttheme{professionalfonts}
\setbeamercovered{transparent}
%NeSI Colors <-------------------------------------------------------------------------------------
\usecolortheme{lily}
\usecolortheme[RGB={47, 68, 71}]{structure} 
\definecolor{nesidark}{HTML}{2F4447}
\definecolor{nesilight}{HTML}{CED9DF}
\definecolor{nesigrey}{gray}{0.7}
\definecolor{nesilightgrey}{gray}{0.98}
\definecolor{nesidarkgrey}{gray}{0.3}
\definecolor{nesiblue}{HTML}{2B9FC2}
\setbeamercolor{block title}{fg=black,bg=nesigrey}
\setbeamercolor{block body}{bg=nesilightgrey,fg=nesidarkgrey}
\setbeamercolor{block body alerted}{bg=white,fg=black}
\setbeamercolor{alerted text}{bg=white,fg=black}
%NeSI Title <---------------------------------------------------------------------------------------
\setbeamerfont{title}{size=\huge}
\frenchspacing
\hyphenation{NeSI}
%NeSI Template parameters <-------------------------------------------------------------------------
\setbeamertemplate{blocks}[default]
\useinnertheme{circles}
\setbeamertemplate{title page}[default][center,rounded=false,shadow=false]
\newcommand\BackgroundPicture[1]{%
\setbeamertemplate{background}{%
\parbox[c][\paperheight]{\paperwidth}{%
\vfill \hfill \includegraphics[height=0.9\paperheight]{#1}
\hfill \vfill
}}}

%Content Starts Here <-------------------------------------------------------------------------------
\title{Title}
%\subtitle{Computational Science team}
\author{Cool Staff Members use \LaTeX{} \\(cool.staff@nesi.org.nz)}
\date{}

\begin{document}

{
\setbeamertemplate{background canvas}{\includegraphics[height=0.99\paperheight]{NeSI_img/Slide00.png}} 
\begin{frame}[plain]
\vspace{1cm}
\titlepage
\end{frame}
}


% This will generate the outline. If you have several topics, uncomment the multicols 
\begin{frame}
\frametitle{Outline}
% \begin{multicols}{2} 
   \tableofcontents
% \end{multicols}
\end{frame}


%%%%%%%%%%%%%%%%%%%%%%%%%%%%%%%%%%%%%%%%%%%%%%%%%%%%%%%%%%%%%%%%%%%%%%%%%%%%%%%%%%%%%%%%%%%%%%%
%%%%%%%%%%%%%%%%%%%%%%%%%%%%%%%%%%%%%%%% Some Examples %%%%%%%%%%%%%%%%%%%%%%%%%%%%%%%%%%%%%%%%%%%
%%%%%%%%%%%%%%%%%%%%%%%%%%%%%%%%%%%%%%%%%%%%%%%%%%%%%%%%%%%%%%%%%%%%%%%%%%%%%%%%%%%%%%%%%%%%%%%

\section{Introduction}
\subsection{About Something}

% Include Graphics <---
\frame[t]
{
  \frametitle{About Something}
\begin{center}
 \includegraphics[width=0.9\textwidth]{NeSI_img/nesi_logo.png}
\end{center}
}

% List with bullets (itemize) <---
% The [<+-| alert@+>] will create a new slide for each element with highlighting 
\frame[t]
{
  \frametitle{About Something}
   \begin{block}{About Something}
   \begin{itemize}%[<+-| alert@+>]
	\item foo
	\item bar
	\item baz
	\item qux
   \end{itemize}
  \end{block}
}

% List with numbers (enumerate) <---
% The [<+-| alert@+>] will create a new slide for each element with highlighting 
\frame[t]
{
  \frametitle{About Something}
   \begin{block}{About Something}
   \begin{enumerate}%[<+-| alert@+>]
	\item inky
	\item pinky
	\item blinky
	\item clyde
   \end{enumerate}
  \end{block}
}

% If you want to include some code (verbatim or listings) you will need to add "fragile" in the frame options <---
\section{Examples}
\subsection{SMP}
\begin{frame}[fragile]
  \frametitle{Submitting a Job}
      \begin{block}{Job Description Example : SMP}
\begin{small}
\begin{verbatim}
#!/bin/bash
#SBATCH -o /share/src/app/example/app-%j.%N.out
#SBATCH -D /share/src/app/example
#SBATCH -J MIGRATESMP
#SBATCH --get-user-env
#SBATCH --ntasks=1
#SBATCH --cpu_bind=core
#SBATCH --export=NONE
#SBATCH --time=08:00:00
ml load migrate/3.4.4-smp
export OMP_NUM_THREADS=4
srun migrate-n-thread parmfile.testml -nomenu
\end{verbatim}
\end{small}
  \end{block}
\end{frame}

\subsection{MPI}
\begin{frame}[fragile]
  \frametitle{Submitting a Job}
     \begin{block}{Job Description Example : MPI}
\begin{small}
\begin{verbatim}
#!/bin/bash
#SBATCH -o /share/src/app/example/app-%j.%N.out
#SBATCH -D /share/src/app/example
#SBATCH -J MIGRATEMPI
#SBATCH --get-user-env
#SBATCH --ntasks=2
#SBATCH --cpu_bind=core
#SBATCH --export=NONE
#SBATCH --time=08:00:00
ml load migrate/3.4.4-ompi
srun migrate-n-mpi parmfile.testml -nomenu
\end{verbatim}
\end{small}
  \end{block}
\end{frame}

\section{example2}
\subsection{Parametric}
\begin{frame}[fragile]
  \frametitle{Submitting a Job}
   \begin{block}{Parametric Job}
To submit 50,000 element job array
\verb|sbatch –array=1-50000 -N1 -i jobin_%a -o jobout_%a job.sh|
Submit time < 1 second\\
\verb|"%a”| in file name mapped to array index\\
Additional environment variable with array index: \verb|SLURM_ARRAY_TASK_ID|\\
 \end{block}
\end{frame}

\begin{frame}[fragile]
  \frametitle{Submitting a Job}
   \begin{block}{Parametric Job}
The management is really easy:
\begin{verbatim}
$ squeue
JOBID PARTITION NODESLIST(REASON) 
123_[2-50000] debug (Resources)
123_1 debug tux0

$ scancel 123_[40000-50000] 

$ scontrol hold 123
\end{verbatim}
 \end{block}
\end{frame}


% End <-----------------------------------------------------------------------------------------
{
\setbeamertemplate{background canvas}{\includegraphics[height=0.99\paperheight]{NeSI_img/Slide00.png}} 
\begin{frame}[plain]
\begin{center}
{\Huge Questions \& Answers}
\end{center}
\end{frame}
}


\end{document} 
